\chapter{Model enhancement}
After having chosen the appropriate machine learning algorithms and found a way to determine the most important features in the classification problem, this chapter is an attempt to create a model able to enhance the results obtained by the basic classification. Three different approaches are presented. First of all, the most important features values were altered and the effect of this modification is examined. Second, a formula for calculating the distance between different samples is established then a way to use that information. Finally, the Hidden Markov Model were used in order to determine the likehood of prediction of a particular class and this data was used to modify the machine learning algorithm. 

\section{Features values modification}
The first proposed solution considers changing the values of most important features in the dataset after the training process. In other words the training process occurs normally, then the importances are determined and a correction function is created (figure \ref{fig:train}). This correction function will act then on the samples introduced to the model in order to change the features values and obtain better predictions (figure \ref{fig:predict}). 

\begin{figure}[H]
    \centering
    \begin{tikzpicture}
        \node[rectangle, draw=black] (main) {Training data};
        \node[rectangle, draw=black] (DecisionTree) [right=of main] {Classifier};
        \node[rectangle, draw=black] (out1) [right=of DecisionTree] {Trained model};
        \node[rectangle, draw=black] (out2) [below=of out1] {Features importances};
        \node[rectangle, draw=black] (feat) [right=of out2] {Features correction model};
   
        \draw[->] (main.east) -- (DecisionTree.west);
        \draw[->] (DecisionTree.east) -- (out1.west);
        \draw[->] (DecisionTree.east) -- (out2.west); 
        \draw[->] (out2.east) -- (feat.west);
    \end{tikzpicture}
    \caption{Training process illustration} \label{fig:train}
\end{figure}

\begin{figure}[H]
    \centering
    \begin{tikzpicture}
        \node[rectangle, draw=black] (main) {Sample};
        \node[rectangle, draw=black] (correl) [right=of main] {Features correction model};
        \node[rectangle, draw=black] (model) [right=of correl] {Trained model};
        \node[rectangle, draw=black](out) [right=of model] {Predicted class};

        \draw[->] (main.east) -- (correl.west);
        \draw[->] (correl.east) -- (model.west);
        \draw[->] (model.east) -- (out.west);
    \end{tikzpicture}
    \caption{Prediction illustration} \label{fig:predict}
\end{figure}

First of all the correction function was defined as the modification function of the feature values for the predicted samples. This modification consists in shifting the feature value so the feature value would not meet the condition to make a false prediction from tables \ref{tab:5best:noev}-\ref{tab:5best:natural}. This function code in Python is as follows:  
\begin{python}
def modify(feat, val):
    X[feat] = X[feat].apply(lambda x: x + val)
\end{python}
where \textit{X} is a pandas DataFrame object containing all the samples to predict.

Second, the five most important features are taken from tables \ref{tab:5best:noev}-\ref{tab:5best:natural} and the values of features from the samples to predict are altered using the previous function. The following code shows this operation:
\begin{python}
modify("R4-PA5:IH", -115.38)
modify("R3-PM2:V", 128525.29)
modify("R2-PM1:V", 2000)
modify("R1-PA12:IH", 32.04)
modify("R3-PM5:I", 330.7)
modify("R3:S", 0)
modify("R2-PA7:VH", 101.20)
modify("R2-PM1:V", -1300872.03)
modify("R3-PA7:VH", 101.22)
modify("R3-PA2:VH", 93.75)    
modify("R2:F", -60)
modify("R3:F", -60)
modify("R2-PA5:IH",- 63.30)
modify("R2-PM7:V", -130857.40) 
\end{python}

The samples modified this way are then used for the predictions. In order to check the success of this method, the classification\_report method from scikit-learn was used. The results before and after modifying the samples are displayed in tables \ref{tab:clreport:before} and \ref{tab:clreport:after}.

\begin{table}[H]
    \centering
    \caption{Classification report before features modification} \label{tab:clreport:before}
    \begin{tabular}{rcccc}\toprule
        & precision    &recall & f1-score  & support \\\midrule
            NoEvents  &   $  0.72 $  &  $ 0.76 $  &  $ 0.74 $  & $ 51797 $\\
              Attack   &  $  0.27 $   & $ 0.26 $  &  $ 0.26 $  & $ 17382 $\\
             Natural   &  $  0.19 $   & $ 0.08 $  &  $ 0.12 $  & $  4232 $\\
            accuracy   &            &          &  $0.60$  &   $73411$ \\
           macro avg   &  $  0.39 $   & $ 0.37 $  &  $ 0.37 $  & $ 73411 $\\
        weighted avg   &  $  0.58 $  &  $ 0.60 $  &  $ 0.59 $ &  $ 73411 $\\\bottomrule
    \end{tabular}
\end{table}

\begin{table}[H]
    \centering
    \caption{Classification report after features modification} \label{tab:clreport:after}   
    \begin{tabular}{rcccc}\toprule
        &precision   & recall & f1-score &  support  \\\midrule

        NoEvents   &    0.71   &   0.83  &    0.77   &  51797 \\
          Attack    &   0.26   &   0.17  &    0.21   &  17382 \\
         Natural   &    0.10   &   0.03   &   0.05  &    4232 \\
    
        accuracy    &          &          &   0.63   &  73411 \\
       macro avg    &   0.36   &   0.34   &   0.34   &  73411 \\
    weighted avg     &  0.57   &   0.63   &   0.59   &  73411    \\     \bottomrule   
    \end{tabular}
\end{table}

The tables \ref{tab:clreport:before} and \ref{tab:clreport:after} show a general increase of the accuracy of the model after changing the features. The more detailed results, show a remarkably better recall and f-measure for NoEvents class, with a small decrease of precision, but for all other classes a decrease can be observed for all the metrics. 

For this particular set of samples, there is a slight increase of weighted recall, weighted f-measure and accuracy. That it is why, it may be concluded that this method succeeded with this particular dataset, especially given the unequal distribution of the samples between classes and model's tendency to predict NoEvents class. However, with any other dataset, where the NoEvents class samples are less common compared to other classes, this method causes worse predictions.

\section{Distance between features}
The second proposed solution considers using a transformation routine, which, reduces the number of features to the 15 most important from tables \ref{tab:5best:noev}-\ref{tab:5best:natural} and adds another feature that represents the closest class to the treated sample. The prediction steps were illustrated on figure \ref{fig:distill}.

\begin{figure}[H]
    \centering
    \begin{tikzpicture}
        \node[rectangle, draw=black](sample) {Sample};
        \node[rectangle, draw=black](trans) [right=of sample] {Transformation};
        \node[rectangle, draw=black](class) [right=of trans] {Classifier};
        \node[rectangle, draw=black](out) [right=of class] {Predicted class};

        \draw[->] (sample.east) -- (trans.west);
        \draw[->] (trans.east) -- (class.west);
        \draw[->] (class.east) -- (out.west);
    \end{tikzpicture}
    \caption{Distance algorithm illustration}
    \label{fig:distill}
\end{figure}

The transformation routine on other hand takes the form of a python class and is composed of 4 methods:
\begin{enumerate}
    \item \textbf{distance(X1, X2)}: returns the distance between two samples X1 and X2. It is calculated as the sum of differences between features,
    \item \textbf{important(X)}: returns the samples with only 15 most important features. The input must be a pandas DataFrame or Series. The choice of features to keep is made by hand directly in the method, without the possibility to change them afterwards,
    \item \textbf{fit(X,y)}: determines the reference class sample for each class by calculating the mean value of each feature among all samples corresponding to the treated class. It is called only during data fitting to the classifier,
    \item \textbf{transform(X)}: determines the class with the smallest distance to the sample, based on reference samples determined by fit(X,y) method. The obtained values are then added as a new feature to the samples X and returned afterwards by the method.
\end{enumerate}

This routine has been coupled with DecisionTree classifier using \textbf{pipeline} class in scikit-learn, which construct acts like a classifier (it has fit and predict methods). 

The success rate of this method was verified, once more, using classification\_report method from scikit-learn. The result before the test are the same as in table \ref{tab:clreport:before}, while the results after using the described method are shown in table \ref{tab:distrep}.

\begin{table}[H]
    \centering
    \caption{Classification report after using the distance routine} \label{tab:distrep}
    \begin{tabular}{rcccc}\toprule
     &   precision    &recall & f1-score &  support  \\\midrule

        NoEvents    &   0.72   &   0.79   &   0.75  &   51797 \\
          Attack    &   0.28   &   0.23   &   0.25  &   17382 \\
         Natural   &    0.19   &   0.08   &   0.11  &    4232 \\
    
        accuracy    &           &         &   0.62   &  73411 \\
       macro avg    &   0.40    &  0.37   &   0.37  &   73411 \\
    weighted avg   &   0.58   &   0.62   &   0.60   &  73411   \\  \bottomrule
    \end{tabular}
\end{table}

Tables \ref{tab:clreport:before} and \ref{tab:distrep} show an increase of model's accuracy by $0.02$. Precision value for Attack class increased by 0.01, with no changes for other classes. Recall value increased for NoEvents by 0.03 and decreased for Attack by 0.03. Finally, f-measure increased for NoEvents class and decreased by 0.01 for other classes. The f-measure value difference for Natural class, despite the same precision and recall, is due to rounding numbers to two decimal places. 

It may be concluded that this method does not really enhance the results, despite the better accuracy. For the other metrics, ones increased, but other decreased, what makes the obtained result less satisfying. In addition to that, the same problem with unequal class distribution mentioned in previous method discussion arises. 

\section{Hidden Markov Model}
The last proposed solution consists in determining the probabilities of occurrence of different classes using so called Hidden Markov Model. The obtained probabilities are then used to modify the split condition in the decision tree classifier. 

The Hidden Markov Model is a probabilistic model for generating observable data in a random way by a sequence of internal hidden states, which can not be observed directly. The transitions between those hidden states have the form of a Markov chain \cite{noauthor_tutorial_nodate}. The Markov chain, on other hand, is a chain of interconnected states, where each state's probability depends only of the probability of predecessing state, without taking into consideration what happened before \cite{amit_introduction_2019}.

In the studied example, the observable data are the features values available in the samples and the states are represented by the 3 classes NoEvents, Attack and Normal. The Hidden Markov Model was used to determine the stationary distribution of the states (invariable in time probability distribution in the Markov chain).

It exists a Python package that implements the Hidden Markov Models and it is compatible with scikit-learn, it's name is \textbf{hmmlearn} \cite{noauthor_hmmlearn_2020}. It can be used just like any other scikit-learn classifiers, especially that it offers methods like fit(), predict(), predict\_proba(), etc...

Using \textbf{hmmlearn} the stationary distribution was determined in 4 lines of code:
\begin{python}
from hmmlearn.hmm import GaussianHMM 
clf2 = GaussianHMM(n_components = 3) # 3 states
clf2.fit(X)
coefs = clf2.get_stationary_distribution()
\end{python}
The Gaussian HMM was used because the observations (features) are continuous. It was initialized by a number of components equal to 3 because of 3 states - NoEvents, Attack and Natural.

The stationary distribution obtained this way was then used to set the class\_weight parameter in DecisionTreeClassifier in scikit-learn, as shown in the listing below.

\begin{python}
ctest = DecisionTreeClassifier(class_weight = {0: coefs[0], 1:coefs[1], 2:coefs[2]})  
\end{python}

The results are presented in table \ref{tab:hmm}.

\begin{table}[H]
    \centering
    \caption{Classification report after using Hidden Markov Model} \label{tab:hmm}
    \begin{tabular}{rcccc}\toprule
       & precision &   recall & f1-score  & support \\\midrule

        NoEvents    &   0.71    &  0.75  &    0.73     &51797\\
          Attack     &  0.27    &  0.26  &    0.26 &   17382\\
         Natural    &   0.16  &    0.06    &  0.09  &    4232\\
    
        accuracy     &          &         &   0.60   &  73411\\
       macro avg     &  0.38    &  0.36   &   0.36  &   73411\\
    weighted avg    &   0.58    &  0.60   &   0.59  &   73411   \\ \bottomrule   
    \end{tabular}
\end{table}

The tables \ref{tab:clreport:before} and \ref{tab:hmm} show that the accuracy did not change after using the probabilities obtained from Hidden Markov Model. Looking more in details, it can be observed a general small decrease of precision and a small increase of recall and f-measure. 

It may be concluded that this method does not really work with the given dataset. The possible cause of this may be the fact that the samples in the dataset does not include any information about their order in time. If the dataset contained information about the time of the event, the results, probably, would be more significant. \\

In this chapter, three methods for results' enhancement were presented. The best result was obtained using the distance between features approach because the values of certain metrics increased without decreasing significally the values of other metrics like in the case of the approach with values modification. The worst results were obtained using the Hidden Markov Model and this probably because the provided dataset is time independent. 