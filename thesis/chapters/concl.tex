\chapter*{Conclusions}
\addcontentsline{toc}{chapter}{Conclusions}

The aim of this thesis was to develop a hybrid neural network capable of differentiating between three states of a cyber physical system - normal behaviour, a fault or an attack. To facilitate the work and focus on computer science topics, ready datasets for a power system was taken and analysed in chapter 2. 

Three different methods were presented to achieve the initial goal - modification of values of features, method using distances between samples and a method using Hidden Markov Model. They were tested on three different machine learning classifiers - Decision Tree, Random Forest and Multilayer Perceptron. 

For Decision Tree classifier the most efficient enhancement method was using the proposed method that works on distances between samples. For Random Forest classifier, the most effective method was modifying the values of features. And for Multilayer Perceptron classifier all the proposed methods failed.

Hidden Markov Model method gave the smallest impact on the tested classifiers. This can be explained by the time independence of the used dataset, especially that this method relies on order of samples.

A possible better solution for the posed problem may be using a physical model of the power system that could estimate on its own some features and combining it with a machine learning technique. There exist some python toolkits created to model a power system like PyPSA \cite{brown_pypsapypsa_2020} or pandapower \cite{noauthor_pandapower_nodate}, but they are complex and require specific advanced technical skills.

This thesis in addition to all that, provides some other important conclusions. First, the different machine learning toolkits do not work always the same way and the results may differ from one toolkit to another, like in this case and the differences in results between Weka and scikit-learn. Second, in order to the results from a work to be reproductible, all the details concerning the used parameters and the version of the software must be provided, in other case it may be hard to get the same result. Finally, it exists plenty of tools for machine learning classifiers evaluation and each toolkit provides its own interpretation of results.

This work may be continued by implementing the power system using one of the mentioned toolkits in order to try to further enhance the results. In addition to that, instead of using scikit-learn, Keras \cite{noauthor_keras_nodate} or PyTorch \cite{noauthor_pytorch_nodate} packages may be used in order to create a complex neural network and possibly get better results.   