\chapter{Conclusions}

The aim of this thesis was to develop a hybrid neural network capable of differentiate between three states of a cyber physical system - normal behaviour, a fault or an attack. To facilitate the work and focus on computer science topics, ready datasets for a power system was taken and analysed in chapter 2. 

The final obtained results do not provide a full answer for the provided aim, because they do not represent in fact hybrid neural network but a decision tree classifier. However, a decision tree classifier is still a machine learning technique and the obtained results may be generalized to the neural networks case.

It was found that the most effective way to enhance classification results was to create a new dataset combining the distance of the samples to the three classes with the values of the fifteen more important features. The other two described method were slightly less effective, especially the method where probabilities found by Hidden Markov Model were used. The inaccuracy of this last method may be explained by the time independence of the used datasets. Maybe using another dataset would ameliorate the results.

The increase of the values of the metrics was not very significant event with the best method. Maybe using a physical model of the power system that could estimate on its own some features and combine this with a machine learning technique would give better results. There exist some python toolkits created to model a power system like PyPSA \cite{brown_pypsapypsa_2020} or pandapower \cite{noauthor_pandapower_nodate}, but they are complex and their use require a specific technical skills.

This thesis in addition to all that, provides some other important conclusions. First, the different machine learning toolkits do not work always the same way and the results may differ from one toolkit to another, like in this case and the differences in results between Weka and scikit-learn. Second, in order the results from a work to be reproductible, all the details concerning the used parameters and the version of the software must be provided, in other case it may be hard to get the same result. Finally, it exists plenty of tools for machine learning classifiers evaluation and each toolkit provides its own interpretation of results.

This work may be continued by implementing the power system using one of the mentioned toolkits in order to try to further enhance the results. In addition to that, instead of using scikit-learn, Keras \cite{noauthor_keras_nodate} or PyTorch \cite{noauthor_pytorch_nodate} packages may be used in order to create a complex neural network and possibly get better results.   