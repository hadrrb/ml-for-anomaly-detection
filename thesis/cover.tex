\newgeometry{top=6cm,bottom=2.2cm,right=0.83cm,left=3.47cm}
\pagestyle{empty}
\pdfoverlaySetPDF{cover.pdf}
\begin{center}

{\large \textbf{PRACA DYPLOMOWA MAGISTERSKA}}
\vspace{6cm}

% {\large \textbf{RAPPORT DE STAGE}}
% \vspace{8cm}

{\fontsize{18}{18}\selectfont \textbf{Hybrid neural networks for anomaly detection in~cyber-physical systems}}
\end{center}
\normalsize

\vspace{6cm}
\noindent\textbf{Wydział Fizyki Technicznej, Informatyki i Matematyki Stosowanej \\
Promotor:} dr hab. inż. Aneta Poniszewska-Marańda\\
\textbf{Dyplomant:} inż. Ramzi Hadrich\\
\textbf{Nr albumu:} 227488\\
\textbf{Kierunek:} Informatyka\\
\textbf{Specjalność:} Computer Science \& Information Technology\\
\begin{center}
    Łódź, \today r.
\end{center}

% \vspace{8cm}
% \noindent Ramzi HADRICH \\
% \textbf{Entreprise:} LORIA\\
% \textbf{Tuteur Entreprise:} Abdelkader LAHMADI\\
% \textbf{Tuteur ENSEM:} Radu RANTA\\
% \textbf{Filière:} ISN\\
% \begin{center}
%     \today
% \end{center}

% First copy: start a new page, and save the page number.
\cleardoublepage
\newgeometry{top=2.2cm, bottom=1.5cm, outer=1cm, inner=2.1cm}
% Uncomment the next line if you do NOT want a page number on your
% abstract and acknowledgments pages.
\begin{abstract}

Nowadays cyber-physical systems are widely used in different application domains. In parallel, machine learning algorithms are used widely to detect the anomalies in the behaviour of these systems. However, this detection is limited to two states: normal behaviour and faulty functioning. This master thesis aims to extend this detection to differentiate between attacks and normal faults. In first place, a power system is described as an example to work on. Then, various machine learning algorithms are evaluated on the given datasets, and this using two machine learning toolkits - scikit-learn and Weka. Later, various tools for feature analysis are presented and an algorithm to find the features that contributed the most into the false predictions is described. Finally, three solutions to the initial problem are presented and evaluated.

\vspace{0.5cm}
\noindent\textbf{Keywords:} machine learning, cyber-physical systems, anomaly detection, random forest, power system
\end{abstract}

% \newpage
% \renewcommand{\abstractname}{Résumé}
% \begin{abstract}
% De nos jours, les systèmes cyber-physiques sont largement utilisés dans différents domaines d'application. Parallèlement, les algorithmes d'apprentissage automatique sont largement utilisés pour détecter les anomalies dans le comportement de ces systèmes. Toutefois, cette détection se limite à deux états : un comportement normal et un fonctionnement défectueux. Ce travail vise à étendre cette détection afin de différencier les attaques des défauts liés au fonctionnement normal. En premier lieu, un système d'alimentation est décrit comme un exemple sur lequel on va travailler. Puis, divers algorithmes d'apprentissage machine sont évalués sur les ensembles de données donnés, et ce à l'aide de deux boîtes à outils d'apprentissage machine - scikit-learn et Weka. Ensuite, différents outils d'analyse des caractéristiques sont présentés et un algorithme permettant de trouver les caractéristiques qui ont le plus contribué aux fausses prédictions est décrit. Enfin, trois solutions au problème initial sont présentées et évaluées - une solution qui utilise les caractéristiques trouvées dans l'étape précédente et qui modifie leurs valeurs, la deuxième qui réduit les échantillons de l'ensemble de données pour n'avoir que les caractéristiques mentionnées précédemment et qui évalue la distance entre ces échantillons et la troisième solution qui utilise le modèle de Markov caché. Les tests ont montré que la solution la plus performante était la seconde, même si le gain de précision n'est pas très significatif.  
% \end{abstract}

\vspace{1cm}
\renewcommand{\abstractname}{Streszczenie}
\begin{abstract}

Obecnie systemy cyber-fizyczne są szeroko wykorzystywane w wielu różnych domenach. Jednocześnie, algorytmy uczenia maszynowego są szeroko stosowane do wykrywania anomalii w działaniu tych systemów. Ale ta detekcja jest ograniczona do dwóch stanów: normalne działanie i naturalne błędy. Ta praca magisterska ma na celu rozszerzenie rozróżniania tych stanów, aby wziąć pod uwagę również ataki na system. W pierwszej kolejności, system energetyczny został opisany jako przykład systemu cyber-fizycznego do dalszych rozważań. Następnie, dokonano oceny różnych algorytmów uczenia maszynowego na bazie zbioru danych z przedstawionego systemu. Do tego celu wykorzystano dwa narzędzia do uczenia maszynowego - scikit-learn i Weka. Później, omówione zostały różne narzędzia do analizy cech systemu. Oprócz tego, zaprezentowano algorytm umożliwiający ustalenie cech najbardziej wpływających na złą klasyfikację danych. Na koniec, trzy rozwiązania pierwotnego problemu zostały przedstawione i sprawdzone. 

\vspace{0.5cm}
\noindent\textbf{Słowa kluczowe:} uczenie maszynowe, system cyber-fizyczny, detekcja anomalii, lasy losowe, system energetyczny
\end{abstract}


% Additional copy: start a new page, and reset the page number.  This way,
% the second copy of the abstract is not counted as separate pages.
% Uncomment the next 6 lines if you need two copies of the abstract
% page.
% \setcounter{page}{\thesavepage}
% \begin{abstractpage}
% \input{abstract}
% \end{abstractpage}

\cleardoublepage

\section*{Preface and acknowledgements}

This thesis was written in the framework of my 5-month internship at LORIA (Laboratoire Lorrain de Recherche en Informatique et ses Applications) in Nancy, France, within my Erasmus+ exchange, from September 2019 to July 2020, at Ecole Nationale Supérieure d'Electricité et de Mécanique (ENSEM), which is a part of University of Lorraine.

Despite COVID-19 pandemic, it was possible to continue and finish the work from home. Mr Abdelkader Lahmadi and Mr Jérome François were there for me during the whole period of the internship. I wanted to say thank you to both of them, for the continuous help, which made this thesis a reality. I wanted also to thank Mr Lahmadi for proposing me this research project.

I am also thankful to the international office team at my home university as well as to my supervisor Mrs Aneta Poniszewska-Marańda for making this Erasmus+ exchange possible.

Finally, I wanted to thank also my family and friends for the support during the writing process of this thesis.
%%%%%%%%%%%%%%%%%%%%%%%%%%%%%%%%%%%%%%%%%%%%%%%%%%%%%%%%%%%%%%%%%%%%%%
% -*-latex-*-
